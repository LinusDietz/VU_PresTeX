%===============================================================================
% Purpose:   Template for \LaTeX beamer presentations
% Created: 
% Autor:  Linus Dietz (linus.dietz@uni-bamberg.de), fork of Marcel Grossmann (marcel.grossmann@uni-bamberg.de)
%===============================================================================

%===============================================================================
% Run pdflatex and bibtex to compile.
%	Configuration in texmaker:  pdflatex -synctex=1 -interaction=nonstopmode %.tex | bibtex % | pdflatex -synctex=1 -interaction=nonstopmode %.tex | pdflatex -synctex=1 -interaction=nonstopmode %.tex
% Edit Information in config/metainfo 
% Choose the language with the following \lang command
% So far only english is supported. TODO Lithuanian
%===============================================================================

% Options: english
\newcommand{\lang}{english}

\documentclass[11pt,\lang ,
%draft,
%handout,
compress
]{beamer}
% Für den Header
% Modify for different languages
\usepackage{ifthen}

\newcommand{\unibastring}{Vilnius University}
%\newcommand{\unibastring}{\ifthenelse{\equal{\lang}{ngerman}}{Universit\"at Bamberg}{University of Bamberg}}

\input{config/commands}


\def\signed #1{{\leavevmode\unskip\nobreak\hfil\penalty50\hskip2em
  \hbox{}\nobreak\hfil(#1)%
  \parfillskip=0pt \finalhyphendemerits=0 \endgraf}}

\newsavebox\mybox
\newenvironment{aquote}[1]
  {\savebox\mybox{#1}\begin{fancyquotes}}
  {\signed{\usebox\mybox}\end{fancyquotes}}


\input{config/hyphenation}

\setbeamertemplate{caption}[numbered]
%\numberwithin{figure}{section}
\begin{document}

\frame{\titlepage}

%\AtBeginSection[]
%{
%  \frame<handout:0>
%  {
%    \frametitle{Outline}
%    \tableofcontents[currentsection,hideallsubsections]
%  }
%}

\AtBeginSubsection[]
{
  \frame<handout:0>
  {
    \frametitle{Outline}
    \tableofcontents[sectionstyle=show/hide,subsectionstyle=show/shaded/hide,subsubsectionstyle=hide]
  }
}

\AtBeginSubsubsection[]
{
  \frame<handout:0>
  {
    \frametitle{Outline}
    \tableofcontents[sectionstyle=show/hide,subsectionstyle=show/shaded/hide,subsubsectionstyle=show/shaded/hide]
  }
}

\newcommand<>{\highlighton}[1]{%
  \alt#2{\structure{#1}}{{#1}}
}

\newcommand{\icon}[1]{\pgfimage[height=1em]{#1}}



\section*{}
\begin{frame}{Content}
\tableofcontents
\end{frame}
%%%%%%%%%%%%%%%%%%%%%%%%%%%%%%%%%%%%%%%%%
%%%%%%%%%% Content starts here %%%%%%%%%%
%%%%%%%%%%%%%%%%%%%%%%%%%%%%%%%%%%%%%%%%%

\section{Logo}
\begin{frame}{Logo}
\framesubtitle{In Blau}
%#1 Breite
%#2 Datei (liegt im image Verzeichnis)
%#3 Beschriftung
%#4 Label fuer Referenzierung
\image{.25\textwidth}{logo.png}{Uni-Logo}{img:logo}
\end{frame}


%%%%%%%%%%%%%%%%%%%%%%%%%%%%%%%%%%%%%%%%%
%%%%%%%%%% References          %%%%%%%%%%
%%%%%%%%%%%%%%%%%%%%%%%%%%%%%%%%%%%%%%%%%
\section*{}
\begin{frame}[allowframebreaks]{References}
\def\newblock{\hskip .11em plus .33em minus .07em}
\scriptsize
\bibliographystyle{IEEEtran}
\bibliography{literature/bib}
\normalsize
\end{frame}




%% Last frame
\frame{
  \vspace{2cm}
  {\huge Questions ?}

  \vspace{20mm}
  \nocite*
  
  \begin{flushright}  
    linus Dietz
    
    \structure{\footnotesize{\href{mailto:linus.dietz@uni-bamberg.de}{linus.dietz@uni-bamberg.de}}}
  \end{flushright}
}


\end{document}
